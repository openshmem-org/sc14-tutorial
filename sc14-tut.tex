\documentclass[10pt,english]{article}
\usepackage[T1]{fontenc}
\usepackage[latin1]{inputenc}
\usepackage{fancyhdr}
\pagestyle{fancy}
\usepackage{babel}
\usepackage{fancybox}
\usepackage[unicode=true]{hyperref}
\usepackage{breakurl}

\usepackage[parfill]{parskip}

\begin{document}

\title{OpenSHMEM Tutorial Proposal for SuperComputing 2014}
\author{Tony Curtis <tonyc@uh.edu> \and Oscar Hernandez <oscar@ornl.gov> \and Pavel Shamis <pshamis@ornl.gov> \and Swaroop Pophale <swaroop@cs.uh.edu>}

\maketitle

\section{Detailed Description}

Programming languages and libraries that leverage sophisticated
hardware capabilities have a natural advantage when used on today's
and tomorrow's high performance computer systems.  This tutorial
introduces the use of OpenSHMEM, a library API that can be used to
write programs that scale with a natural expression of 1-sided,
overlapped communication and computation.

The overarching goal is to present the capabilities of the OpenSHMEM
library to the audience with a view to driving the development of the
library through community involvement.  The bigger the user community,
the more input there will be to help decide future directions.

The OpenSHMEM project is an effort to develop the current API to make
it more expressive on current and future HPC systems.  The effort
includes development of the specification of OpenSHMEM, the web site,
tutorial, presentation, workshop and other outreach materials, along
with new tools that programmers can use to detect and avoid problems
in OpenSHMEM programs at compile-time.

People who attend this tutorial will find out about the history of
OpenSHMEM, its current state, and where we see it going.  Co-design of
software and hardware is a hot topic as we try to build ever larger
systems without running into power and financial brick walls.

Designing expressive programming environments that allow programmers
to solve problems is part of the solution, but also if we can closely
match the anticipated hardware capabilities, especially within network
fabrics, we can build yet more efficient systems.  Communication
layers that can take advantage of existing and novel network hardware
will allow the development of yet more interesting higher-level
environments such as libraries like OpenSHMEM and highly-productive
languages.  Oak Ridge National Laboratory has been developing one such
interface called the Uniform Common Communication Substrate (UCCS);
part of the tutorial will cover the current state of this library and
how we are working to provide a very high performing software
interface to networks.

Our intended audience are programmers and technology decision makers
who are looking for information to help them decide on future HPC
development methods.  The tutorial will be gradated to be attractive
to programmers who are unfamiliar with the details of OpenSHMEM and
PGAS environments, as well as more seasoned programmers and
technologists who want to acquaint themselves with the state of the
art.

The tutorial is intended for people who can program in C and/or
Fortran, or at least are familiar with such code, as those are the
languages that OpenSHMEM targets, but note that we will concentrate on
C for our examples.  A familiarity with parallel programming
environments will be be helpful, e.g. GNU/Linux usage, MPI, clusters
(job schedulers, parallel launchers).

\subsection{General Description}

The tutorial covers introductory material, going over the history of
OpenSHMEM to provide context for the current efforts, and discusses
the current state of play, before looking at complementary efforts for
network support and the tool ecosystem that will make OpenSHMEM a
highly productive environment.  We look at some example programs that
demonstrate how OpenSHMEM works, and will make available code
examples.  Installation and use of the library will be covered for
those who wish to participate in development.  Information about how
to propose new features and take part in online discussions will be
given so that attendees can continue to involve themselves in the
process after the tutorial.

\subsection{Cohesion of Presenters}

The tutorial will be presented by researchers from the University of
Houston (UH), and Oak Ridge National Laboratory (ORNL).  These 2
institutions have worked together for a number of years on a number of
projects.  As well as the 4-year OpenSHMEM effort, UH graduate
students have been perennial summer interns at ORNL over the past few
years, and there are UH post-doctorates working at ORNL, including one
of the presenters.  The development of the tutorial material thus
comes from a longstanding collaborative process, and we have given
joint tutorials and workshops in the past, including this year.

\subsection{Update Of Previous Material}

The material we intend to present has in part appeared before in
previous tutorials.  This is to aid in the familiarization process as
many attendees are new to OpenSHMEM.  To add to existing material, we
include some case studies of programs that were either ported from
other paradigms, e.g. MPI, or written from scratch, and show how to
progressively improve the parallelization afforded by OpenSHMEM.

\section{Detailed Outline}

\subsection{Overview}

  To set context, we will provide the audience with an outline of the
  space occupied by OpenSHMEM in the programming landscape, comparing
  it with other libraries and languages in the Partitioned Global
  Address Space family.  To show how the API is structured, we will
  take a guided tour of the API, highlighting with examples how
  OpenSHMEM is used in programs.

\subsection{The OpenSHMEM API}

  The API consists of 5 basic sections: memory management,
  point-to-point data transfers, synchronizations, collective
  operations, and atomics and locks.  Introduced in this order, we can
  incrementally describe working OpenSHMEM programs that grow in
  complexity.

\subsection{UCCS Development}

  We will discuss the current state and future plans of the UCCS
  substrate software.  Design decisions will be be covered with a view
  to explaining how the software is intended to be a platform not just
  for OpenSHMEM, but also for PGAS/HPCS languages.

\subsection{OpenSHMEM is evolving}

  We will present existing extension ideas (both at the time of
  writing and those in play at the time of the tutorial) and draw on
  the experience of the audience to actively solicit new ideas.

\subsection{Debugging and Profiling Tools}

  OpenSHMEM is supported by various well-known profiling and debugging
  tools, including TAU, Vampir, and DDT.  We will discuss their use
  briefly, but avoid in-depth discussion because other tutorials are
  likely to be focused on these topics specifically.

\subsection{Our new OpenSHMEM Analyzer tool}

  This will be covered to help attendees understand how to detect
  errors and inefficiencies in programs.  Examples will show how the
  tool is able to report on under-synchronized programs which can lead
  to race conditions, hangs or undefined behavior, and
  over-synchronized ones which introduce slowdowns.
  
\subsection {Application Experience with OpenSHMEM}
On this section we will talk about different applications and bnechmarks that
have been ported to OpenSHMEM and how these applications benefit
from the communication library. The applications will include WL-LSMS3, 
SSCA3,  NPB, QMCPack, etc. We will explain why these applications benefit
from OpenSHMEM.

\section{Hands-On Work}

The OpenSHMEM reference implementation and its dependencies are
available in source form for download.  The library includes demo
programs that attendees can examine and work with.  The software runs
both on a desktop/laptop and on larger clusters, giving a wide range
of deployment options for this tutorial.  For those interested in
obtaining the software themselves, we will provide assistance in
installing from source on GNU/Linux.  Hands-on sessions in this
tutorial will allow attendees to try things out at their own pace so
these activities can occur concurrently with the presenters available
to help.

We will walk attendees through installing the OpenSHMEM reference
implementation and GASNet, and then show how to use the library in a
number of configurations, including in a personal laptop and on a
cluster with Infiniband interconnect.  There will be discussion of
other common combinations.

The demonstration programs that come with the library, plus some other
sample programs are available for building and experimentation.
Having working examples helps people to see how to write their own
programs from scratch.

Some MPI programs with OpenSHMEM counterparts will be available to
show how existing parallel programs can be converted.

All tutorial material will be available for download.

\section{R\'{e}sum\'{e}s}

\subsection{Tony Curtis}

    I am a senior member of Dr. Barbara Chapman's HPCTools research
    group in Computer Science.  My main work is lead on the OpenSHMEM
    project with 2 post-docs and 3 PhD students, as well as leading
    system administration of group resources.  I provide a consultant
    role in the university's High Performance Computing Center, of
    which Barbara Chapman is also director.

    For the OpenSHMEM project, we have presented workshops/tutorials
    at various conferences, for example PGAS 2011 and 2012, and
    members of the team have presented posters and papers at other
    conferences.  I have also led a number of OpenSHMEM BOFs at SC
    conferences.

\subsection{Oscar Hernandez}

    Oscar Hernandez is a research staff member of the Computer Science
    and Mathematics Division at Oak Ridge National Laboratory. He
    works on code analysis and transformation tools to support the
    NCCS and OLCF project activities. His research focus has been on
    compilers and performance tools integration, and optimization
    techniques for parallel languages and libraries, especially
    OpenSHMEM, OpenMP, OpenACC and other accelerator directives. His
    work also involves the research and development of tools for
    OpenSHMEM programs and is the main developer of the OpenSHMEM
    Analyzer. Prior to this work, he was a developer of the OpenUH
    compiler and the Dragon Analysis Tool to analyze OpenMP
    programs. Oscar Hernandez graduated from the University of Houston
    with a Ph.D. and Msc. degree in the area of compilers and high
    performance computing. He also holds a B.S.  Degree in Physics,
    Mathematics and Computer Science from Harding University.

\subsection{Pavel Shamis}

    Pavel Shamis is a research staff member at Oak Ridge National
    Laboratory.  His main area of research is high-performance
    communication middleware and collective communication. Prior to
    joining Oak Ridge National Laboratory, Mr. Shamis spent ten years
    at Mellanox Technologies in different technical roles, including
    Senior Software Developer and HPC Team Leader.

\subsection{Swaroop Pophale}

   Swaroop is a doctoral candidate in the Computer Science Department
   at the University of Houston. She has been involved in high
   performance computing research and the OpenSHMEM project for the
   last four years. Her main area of research is collective
   optimizations and compiler-based analysis for collectives. Swaroop
   was one of the main drivers for the OpenSHMEM 1.1 specification and
   has vast experience in porting codes to OpenSHMEM.

\end{document}
